% v2-acmlarge-sample.tex, dated March 6 2012
% This is a sample file for ACM large trim journals
%
% Compilation using 'acmlarge.cls' - version 1.3, Aptara Inc.
% (c) 2011 Association for Computing Machinery (ACM)
%
% Questions/Suggestions/Feedback should be addressed to => "acmtexsupport@aptaracorp.com".
% Users can also go through the FAQs available on the journal's submission webpage.
%
% Steps to compile: latex, bibtex, latex latex
%
%\documentclass[prodmode,acmtap]{acmlarge}
\documentclass[10.5pt]{article}


%\usepackage[a4paper,pdftex]{geometry}
\usepackage[utf8x]{inputenc}
\usepackage[italian]{babel}
\usepackage{makeidx}
\usepackage{graphicx}

\linespread{1.1}

% Package to generate and customize Algorithm as per ACM style
%\usepackage[ruled]{algorithm2e}
%\SetAlFnt{\algofont}
%\SetAlCapFnt{\algofont}
%\SetAlCapNameFnt{\algofont}
%\SetAlCapHSkip{0pt}
%\IncMargin{-\parindent}
%\renewcommand{\algorithmcfname}{ALGORITHM}

% Title portion
\title{RoundWord\\
\large{Progetto per il Corso di Sistemi Distribuiti, a.a. 2012-2013}}
\author{MATTEO BRUCATO e MIRO MANNINO\\Università di Bologna}


\begin{document}
%\large

\maketitle

\begin{abstract}
...
\vspace*{100mm}
\end{abstract}


\tableofcontents

\section{Introduzione}
% in cui si inquadra il problema affrontato, chiarendo gli obiettivi, riassumendo lo stato dell'arte, e descrivendo la struttura della relazione.


\section{Aspetti progettuali}
% in cui si illustra il progetto svolto; in particolare si discutono i problemi specifici affrontati, le soluzioni valutate e proposte, e l'architettura astratta del sistema sviluppato.


\section{Aspetti implementativi}
% dettagli sulle scelte implementative, ed architettura specifica implementata. Inserire almeno il diagramma delle classi e uno delle interazioni secondo lo standard UML.


\section{Valutazione}
% confronto delle soluzioni proposte con soluzioni analoghe allo stato dell'arte.


\section{Conclusioni}
% commenti conclusivi su possibili miglioramenti di quanto discusso, e possibili linee di intervento futuro.




% Bibliography
\bibliographystyle{ACM-Reference-Format-Journals}
\bibliography{riferimenti}


\end{document}
% End of v2-acmlarge-sample.tex (March 2012) - Gerry Murray, ACM
